\documentclass[article,12pt,onesidea,4paper,english,brazil]{abntex2}

\usepackage{lmodern, indentfirst, nomencl, color, graphicx, microtype, lipsum}			
\usepackage[T1]{fontenc}		
\usepackage[utf8]{inputenc}		

\setlrmarginsandblock{2cm}{2cm}{*}
\setulmarginsandblock{2cm}{2cm}{*}
\checkandfixthelayout

\setlength{\parindent}{1.3cm}
\setlength{\parskip}{0.2cm}

\SingleSpacing

\begin{document}
	
	\selectlanguage{brazil}
	
	\frenchspacing 
	
	\begin{center}
		\LARGE FOTOS VIVAS SOBRE A TORTURA NO PERÍODO\\DA DITADURA MILITAR\footnote{Trabalho realizado dentro (da área de Conhecimento CNPq/CAPES: Ciências Humanas).}
		
		\normalsize
		André Luís Monteiro Ferreira Lopes\footnote{andre.monteiro@ifro.edu.br, Campus Colorado do Oeste.} 
	\end{center}
	
	\noindent A ditadura civil-militar (1964-1985) foi um período onde as liberdades de expressão, opinião e de pensamento político foram controladas pelos militares, muitos mecanismos foram utilizados para cercear esses direitos, dentre os quais podemos citar os Atos Institucionais, a censura dos meios de comunicação e a tortura. Durante a ditadura militar a tortura passou a ser institucionalizada em 1968, através do Ato Institucional número 5 (A.I. -5) a tortura passou a ser aplicada sistematicamente, aos opositores políticos do regime. Além disso, a tortura durante a ditadura teve como principal papel o controle social. Este estudo teve como propósito a reflexão sobre a utilização de torturas físicas e psicológicas durante a ditadura militar no Brasil, um dos períodos mais violentos da história de acordo com a historiografia escrita para esse período histórico. Através desse relevante pode-se perceber que ainda hoje, mais de 50 anos depois do golpe militar, ainda há lacunas que precisam ser esclarecidas. Nesse sentido buscamos através de pesquisas bibliográficas qualitativas entender e qualificar a tortura utilizada durante o período militar, identificar os principais tipos de torturas utilizados bem como os impactos causados por esse tipo de violência a sociedade, e qualificar a tortura como uma conduta extremamente danosa parar um grande número de famílias e a sociedade brasileira. Como resultado desta análise bibliográfica foi possível expor para a comunidade escolar por meio da técnica artística denominada “Foto Viva”, onde os discentes se caracterizaram com indumentárias, gestos e feições de torturadores e vítimas num processo interpretativo de representar as torturas como imagens vivas com o interesse de impactar os observadores.
	\vspace{\onelineskip}
	
	\noindent
	\textbf{Palavras-chave}: Tortura. Violência. Ditadura Militar.
	
\end{document}
