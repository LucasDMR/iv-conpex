\documentclass[article,12pt,onesidea,4paper,english,brazil]{abntex2}

\usepackage{lmodern, indentfirst, color, graphicx, microtype, lipsum}			
\usepackage[T1]{fontenc}		
\usepackage[utf8]{inputenc}		

\setlrmarginsandblock{2cm}{2cm}{*}
\setulmarginsandblock{2cm}{2cm}{*}
\checkandfixthelayout

\setlength{\parindent}{1.3cm}
\setlength{\parskip}{0.2cm}

\SingleSpacing

\begin{document}
	
	\selectlanguage{brazil}
	
	\frenchspacing 
	
	\begin{center}
		\LARGE SEUS OLHOS SÃO MEUS OLHOS\footnote{Trabalho realizado na Área: Educação. Com financiamento para publicação do IFRO Campus Vilhena.}
		
		\normalsize
		Edson Gomes Marinho Júnior\footnote{Coordenador, edson.junior@ifro.edu.br;Campus Colorado do Oeste.} 
		Tatiana de Abreu Curado Rezende\footnote{Colaboradora, tatiana.rezende@ifro.edu.br, Campus Vilhena.}
		Joacir Aparecido Lourenzoni\footnote{Colaborador; joacir.lourenzoni@ifro.edu.br;Campus Colorado do Oeste.} 
	\end{center}
	
	\noindent A Educação Inclusiva tem como referência o direito à diversidade, o sentimento de aceitação diante do diferente e do conhecimento a respeito das deficiências físicas. No caso da deficiência visual, a falta de informação também provoca dificuldades em relação ao diálogo entre pessoas videntes e cegas, já que estas últimas tem uma percepção de mundo que vai além do sentido da visão, transpassando por todos os demais sentidos. O processo de inclusão escolar requer o desenvolvimento da empatia por parte dos envolvidos na rotina educacional, para que seja possível compreender que as diferenças existem em todos, que todos são diferentes entre si, possuindo suas particularidades, dificuldades e medos. Este projeto foi desenvolvido com o intuito de construir o hábito de “se colocar no lugar do outro”, através da aplicação de atividades adaptadas para deficientes visuais aos alunos dos 2º Anos do Curso Técnico em Agropecuária Integrado ao Ensino Médio do IFRO – Campus Colorado do Oeste. Durante as aulas de Educação Física, os alunos foram divididos em duplas, nas quais ocorreu o revezamento do uso de uma venda nos olhos, para que um integrante fosse o “deficiente visual” e o outro o “vidente”. Desta forma, realizaram caminhadas pelo ginásio de esportes do Campus, passando por obstáculos como as traves dos gols e alambrados, também participaram dos seguintes jogos para-desportivos para cegos: Goalball, Futebol de Cinco e Salto em Distância.  Complementando a vivência, durante um mês as duplas almoçaram no refeitório do Campus, com um deles vendado enquanto o colega da dupla guiava e a cada dia os papéis se invertiam. De forma unânime, os adolescentes relataram que esta experiência desenvolveu neles a capacidade de lidar com adversidades e dificuldades, além de promover a autoconfiança melhora no relacionamento interpessoal, construindo valores de respeito, paciência e empatia para com todas as diversidades e deficiências físicas.
	
	\vspace{\onelineskip}
	
	\noindent
	\textbf{Palavras-chave}: Educação Inclusiva. Deficiência visual. Ensino Médio;
	
\end{document}
