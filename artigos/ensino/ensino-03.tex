\documentclass[article,12pt,onesidea,4paper,english,brazil]{abntex2}

\usepackage{lmodern, indentfirst, nomencl, color, graphicx, microtype, lipsum}			
\usepackage[T1]{fontenc}		
\usepackage[utf8]{inputenc}		

\setlrmarginsandblock{3cm}{3cm}{*}
\setulmarginsandblock{3cm}{3cm}{*}
\checkandfixthelayout

\setlength{\parindent}{1.3cm}
\setlength{\parskip}{0.2cm}

\SingleSpacing

\begin{document}
	
	\selectlanguage{brazil}
	
	\frenchspacing 
	
	\begin{center}
		\LARGE APRENDENDO GEOMETRIA ANALÍTICA COM O GEOGEBRA.
		
		\normalsize
    	Antônio Sérgio F. dos Santos\footnote{Antônio.santos@ifro.edu.br, Campus Vilhena.} 
		Ederson da Silva Bressanini\footnote{Colaborador.} 
	\end{center}
	
	\noindent O ensino da Matemática nos últimos anos vem enfrentando grandes dificuldades entre nossos jovens. Segundo o relatório De Olho nas Metas, divulgado no dia 06 de março de 2013, apenas 10,3\% dos jovens brasileiros têm aprendizado adequado em matemática ao final do ensino médio. Os dados têm como base os resultados da Prova Brasil/Saeb 2011. Esse é o pior índice, desde que o programa Todos Pela Educação (movimento da sociedade civil brasileira que tem como missão contribuir para que até 2022 o país assegure a todas as crianças e jovens o direito de Educação Básica de qualidade, fundada em 2006), começou a monitorar o indicador. Diante da grande dificuldade no ensino da Matemática, buscamos com este trabalho mostrar caminhos visando propiciar, ao ensino da Matemática, aulas atrativas e dinâmicas. Este estudo é voltado à utilização dos softwares matemáticos, em especial o Geogebra, como recurso no processo ensino aprendizagem da Matemática, no qual os alunos tiveram a oportunidade de comparar o ensino tradicional com um ensino dinâmico e participativo, em que os mesmos foram levados a criar, explorando seu raciocínio e criatividade. O objetivo principal desse estudo foi proporcionar novos caminhos na busca de um ensino de qualidade, voltado para o aluno, capaz de superar as barreiras naturais e históricas, no ensino aprendizagem da matemática. O uso do computador nas aulas de Matemática tem tomado espaço cada vez mais importante, considerando que nossos alunos vivem rodeados das Tecnologias de Informações e Comunicações, que são as tecnologias usadas como meios de comunicação, como equipamentos de informática, celulares, sites de web, software, etc.
	
	\vspace{\onelineskip}
	
	\noindent
	\textbf{Palavras-chave}: Softwares matemáticos. Ensino aprendizagem. Geogebra.
	
\end{document}
