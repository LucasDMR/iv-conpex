\documentclass[article,12pt,onesidea,4paper,english,brazil]{abntex2}

\usepackage{lmodern, indentfirst, nomencl, color, graphicx, microtype, lipsum,textcomp,multirow}			
\usepackage[T1]{fontenc}		
\usepackage[utf8]{inputenc}		

\setlrmarginsandblock{2cm}{2cm}{*}
\setulmarginsandblock{2cm}{2cm}{*}
\checkandfixthelayout

\setlength{\parindent}{1.3cm}
\setlength{\parskip}{0.2cm}

\SingleSpacing

\begin{document}
	
	\selectlanguage{brazil}
	
	\frenchspacing 
	
	\begin{center}
		\LARGE PRODUÇÃO DE VINHO A PARTIR DE CAJÁ-MANGA (\textit{SPONDIAS DULCIS})
		
		\normalsize
		Juliana Alves Rodrigues\footnote{Bolsista de Iniciação Cientifica Junior – PIP IFRO, juliana2013ifro@gmail.com, IFRO Ji-Paraná.} 
		José Antonio Avelar Baptista\footnote{Co-orientador, jose.antonio@ifro.edu.br IFRO Ji-Paraná.} 
		Renato Ándre Zan\footnote{Orientador, Professor EBTT Química, renato.zan@ifro.edu.br IFRO Ji-Paraná.}  
	\end{center}
	
	% resumo em português
	\begin{resumoumacoluna}
		A fabricação e consumo de bebidas alcoólicas é uma das atividades mais antigas desenvolvidas pelo homem, a produção de vinho iniciou-se por volta de
		7.000 a.C. segundo relatos. A bebida foi sendo desenvolvida paralelamente aos processos de fermentação de cereais. O vinho no Brasil é reconhecido mundialmente por sua qualidade. O objetivo da pesquisa buscou-se incialmente procurar um fruto que pudesse gerar todas as qualidades possíveis o fruto escolhido sendo cajá-manga, através do fruto pode se avaliar algumas características físico- químicas da bebida produzida e potencializando a possibilidade de um novo produto para registro de patente, através desta pesquisa possibilitou verificar a qualidade do fruto e do vinho, produziu-se o vinho seco e suave.
		
		\vspace{\onelineskip}
		
		\noindent
		\textbf{Palavras-chave}: ruto.bebida.qualidade.
	\end{resumoumacoluna}
	
	\textual
	
	\section*{Introdução}
	
	maioria do vinho produzido no Brasil vem da região Sul, chegando a cerca de 95\% dos vinhos brasileiros serem industrializados no Rio Grande do Sul (IBRAVIN, 2010). No que se diz respeito ao número de frutas exóticas existentes na região amazônica, fica-se a dúvida se as mesmas podem ser usadas na produção de produtos comerciais como vinho artesanal, e que muitas bebidas nesta área tem surgido com as mais variadas espécies de frutas. A utilização do fruto cajá-manga verificou a eficácia para a fermentação.
	Uma vez que a indústria de bebidas tem se voltado a busca de novos produtos para a diversificação de seus comércios, esta pesquisa se justifica por buscar nova possibilidade de produção de vinho através de frutas nativas da região amazônica, desenvolvendo novos produtos, e podendo assim também dar um direcionamento a frutas muitas vezes pouco conhecidas e pouco comercializadas também da região, surgindo assim uma nova possibilidade de renda para pequenos agricultores da região.
	
	\section*{Material e Método}
	
	Antes do manuseio de todo os objetos cada equipamento foi submetido a um processo de esterilização, no intuito de evitar a contaminação do produto por microrganismos presentes nos recipientes. Os materiais que foram utilizados são: Panela, chapa aquecedora, béquer, balde com lacre e garrafas de vidro.
	
	\subsection*{Obtenção do mosto}
	
	Os materiais utilizados para realização dos ensaios fermentativos serão higienizados com álcool 70\%. A massa de cajá-manga necessária para cada fermentação será adquirida na forma de polpa ou fruta no comercio local, será descongelada a temperatura ambiente quando polpa , ou despolpada e triturada em um processador quando na forma de fruta. Para obtenção de um vinho resultante com aromas delicados e refinados, filtrou o suco, obtendo um mosto sem fibras.
	
\subsection*{Preparação do Inóculo}
	
	Para a preparação do inóculo usou-se o fermento livre de outros microrganismos (antagônicos e competidores) cultivado em uma quantidade de mosto inicial, em alta taxa de multiplicação para a adaptação das leveduras - foi retirada uma alíquota de 20\% do total a ser fermentado, que foi fervida em banho- maria durante 15 minutos e em seguida resfriada a temperatura ambiente; o restante do mosto foi devidamente refrigerado.
	
	\subsection*{Fermentação Alcoólica}
	
	Após 24 horas adicionou- se a outra fração do mosto ao fermentador, ajustou-se com sacarose para obtenção de um maior grau alcoólico e pH adequado. Durante o andamento da fermentação foram realizadas análises em amostras de 10 ml até que ocorresse a estabilização do ºBrix. O produto foi submetido a um choque térmico para cessar a atividade das leveduras, e então armazenou-se por alguns dias em local resfriado e com proteção contra a luz. Isso garantirá a sedimentação de sólidos, decorrentes da fermentação. O vinho límpido utilizou-se em diversas análises a fim de caracterizar o fermentado a ser produzido. Essas
	análises nos permitiram concluir sobre a qualidade de aspectos do vinho produzido e verificou-se o enquadramento dele na legislação brasileira.
	
\subsection*{Processo de produção do vinho cajá-manga suave}

	Utilizou-se o liquidificador para triturar 2 kg de polpa de cajá manga no intuito de obter o suco do fruto. Em seguida, o líquido obtido foi transferido para a panela e posto para aquecer com 3 litros e 200 ml de água destilada, enquanto aquecia até chegar o ponto de fervura, foi colocado em um béquer de 2 litros, 500ml de água destilada e aquecido até chegar em torno de 60° e foi adicionado aos poucos o açúcar até chegar a 1,5kg e mexendo constantemente até obter uma calda. Em seguida a calda foi adicionada aos poucos no liquido que estava aquecendo na panela até obter 25°brix e deixado ferver por 4 minutos.
	
\subsection*{Fermentação do mosto de cajá-manga suave}
	
	Quando o liquido encontrou-se no ponto ideal de fervura (4 minutos), o conteúdo foi novamente transferido, porém dessa vez para um balde especifico, adicionando 12g de fermento (saccharomyces cerevisiae). O recipiente foi totalmente lacrado, com exceção de uma torneira na parte superior na qual foi colocou-se uma mangueira direcionada a um Becker com água. Deixando fermentar por 30 dias e agitando o balde nos três primeiros dias.
	
	\subsection*{Finalização do processo do vinho cajá-manga suave}
	
	Ao todo, o vinho permaneceu 30 (trinta) dias fermentando e no término do prazo todo o liquido foi coado em torno de 3 vezes para separar o liquido de qualquer resíduo mais viscoso presente no conteúdo. Por último, a bebida foi adicionadas nas garrafas e armazenadas como recomendado.
	
	\subsection*{Processo de produção do vinho cajá-manga seco}
	
	Utilizou-se o liquidificador para triturar 2 kg de polpa de cajá manga no intuito de obter o suco do fruto. Em seguida, o líquido obtido foi transferido para a panela e posto para aquecer com 3 litros e 200ml de água destilada, enquanto aquecia, foi colocado em um béquer de 2 litros, 500ml de água destilada e aquecido até chegar em torno de 60° e foi adicionado aos poucos o açúcar até chegar a 1,5kg e mexendo constantemente até obter uma calda. Em seguida a calda foi adicionada aos poucos no liquido que estava aquecendo na panela até obter 16°brix e deixado ferver por 4 minutos.
	
	\subsection*{Fermentação do mosto de cajá-manga seco}
	
	Quando o liquido encontrou-se no ponto ideal de fervura (4 minutos), o conteúdo foi novamente transferido, porém dessa vez para um balde especifico, adicionando 12g de fermento (saccharomyces cerevisiae). O recipiente foi totalmente lacrado, com exceção de uma torneira na parte superior na qual foi colocada uma mangueira direcionada a um Becker com água. Deixando fermentar por 30 dias e agitando o balde nos 3.
	
	\subsection*{Finalização do vinho cajá-manga seco}
	
	Ao todo, o vinho permaneceu 30 (trinta) dias fermentando e no término do prazo todo o liquido foi coado em torno de 3 vezes para separar o liquido de qualquer resíduo mais viscoso presente no conteúdo. Por último, a bebida foi adicionadas nas garrafas e armazenadas como recomendado.
	
	\section*{Resultados e Discussão}
	
Com base na tabela 1 abaixo mostrando os valores iniciais e finais, das analises físico-químicas feita com o vinho, sendo teor de Brix, pH, densidade e temperatura, que variam conforme a fermentação do mosto, seco e suave do fruto cajá-manga. Observa-se que há uma grande variação do Brix inicial e final do vinho seco, que se dá pela fermentação entra em decréscimos no grau de Brix e a densidade durante a fermentação, verificando-se o baixo teor açúcar no vinho seco o diferente do vinho suave no saber. O grau de Brix observado no vinho suave 28° inicial, com uma grande elevação no açúcar do mosto que decresceu durante o período de fermentação finalizando no grau de Brix 15°.

\begin{table}[h]
	\centering
	\caption{Valores das analises físico-químicas dos fermentados suave e seco.}
	\label{my-label}
	\begin{tabular}{|l|l|l|l|l|l|l|l|l|}
		\hline
		\multirow{2}{*}{Vinho} & \multicolumn{2}{c|}{ºBrix} & \multicolumn{2}{c|}{pH} & \multicolumn{2}{c|}{Densidade} & \multicolumn{2}{c|}{Temperatura} \\ \cline{2-9} 
		& (inicial)     & (final)    & (inicial)   & (final)   & (inicial)       & (final)      & °C (inicial)     & °C (final)    \\ \hline
		Seco                   & 15°           & 5°         & 3,02        & 3,09      & 1.1002          & 1.0240       & 25°C             & 24°C          \\ \hline
		suave                  & 28°           & 15°        & 3,14        & 3,12      & 1.1017          & 0.8486       & 25,8°C           & 23,8°C        \\ \hline
	\end{tabular}
\end{table}

	
	\section*{Conclusões}
	
Com base na pesquisa, conclui-se que a escolha do fruto cajá-manga foi eficaz para o preparo do vinho seco e suave, proporcionando um maior rendimento ao longo das analises físico-químicas de maneira correta e seguindo os paramentos para um resultado excelente na produção do vinho com um sabor e qualidade esperado.
	
	\section*{Instituição de Fomento}
	
	Instituto Federal de Rondônia campus Ji-Paraná.
	
	\section*{Referências}
	
	\sloppy
	
\noindent LIMA, U. A. Aguardente: fabricação em pequenas destilarias. Piracicaba: Fealq, 1999.

\noindent LIMA, U. A. Produção de etanol. In: AQUARONE, E.; BORZANI, W., SCHMIDELL,
W.; LIMA, U. A. Biotecnologia: tecnologia das fermentações. Vol. 1, p. 48-69. São Paulo: Edgard Blücher BINAGRI, 1975. Disponível em: http://orton.catie.ac.cr/cgi-bin/wxis.exe/. Acesso em 25 out. 2012.

\noindent LOPES, C. H.; GABRIEL, A. V. M. D. Tecnologia de produção de etanol. São Carlos: EDUFSCAR, 2010. No prelo.

\noindent MUNIZ, C. R. et al. Bebidas fermentadas a partir de frutos tropicais. Boletim do Centro de Pesquisa e Processamento de Alimentos, v. 20, n. 2, p. 309–322, 2002.

\noindent RIZZON, L. A.; ZANUZ, M. C.; MANFREDINI, S.; Como Elaborar Vinho de Qualidade na Pequena Propriedade, 3a ed., Embrapa: Bento Gonçalves, 1994.
	
\end{document}
