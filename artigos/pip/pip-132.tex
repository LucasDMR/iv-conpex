\documentclass[article,12pt,onesidea,4paper,english,brazil]{abntex2}

\usepackage{lmodern, indentfirst, nomencl, color, graphicx, microtype, lipsum, placeins, longtable,caption}			
\usepackage[T1]{fontenc}		
\usepackage[utf8]{inputenc}		

\setlrmarginsandblock{2cm}{2cm}{*}
\setulmarginsandblock{2cm}{2cm}{*}
\checkandfixthelayout

\setlength{\parindent}{1.3cm}
\setlength{\parskip}{0.2cm}

\SingleSpacing

\begin{document}
	
	\selectlanguage{brazil}
	
	\frenchspacing 
	
	\begin{center}
		\LARGE OS BENEFÍCIOS PEDAGÓGICOS DO JOGO DE XADREZ\footnote{TrabalhorealizadodentrodaCiênciasdasaúde/EducaçãofísicacomfinanciamentodoIFRO–Campus Ariquemes.}
		
		\normalsize
		Leonardo Castilho Cardozo\footnote{Bolsista (PIBIC-ET), leonardocastilho1990@hotmail.com, Campus Ariquemes.} 
		Henrique Dalle Luque Barbosa dos Santos\footnote{Bolsista (PIBIC-ET), henriqueluque2506@gmail.com, Campus Ariquemes.} 
		Francisco Roberto da Silva de Carvalho\footnote{Orientador, francisco.roberto@ifro.edu.br, Campus Ariquemes.} 
		Alexandre Thomé da Silva de Almeida5\footnote{Co-orientador, alexandre.thome@ifro.edu.br, Campus Ariquemes.} 
	\end{center}
	
	% resumo em português
	\begin{resumoumacoluna}
		O presente projeto teve como objetivo pesquisar os benefícios do jogo de xadrez para os alunos que o praticam regulamente. Tratou-se de uma revisão bibliográfica que foi pesquisada em livros, teses, dissertações e artigos sobre a temática, procurando meios de fortalecer, intensificar e embasar a ideia da utilização do jogo de xadrez em sala de aula, seja como instrumento pedagógico, seja como atividade recreativa. Este trabalho buscou demostrar à importância da pratica do xadrez no desenvolvimento cognitivo dos alunos que a praticam, verificado a possibilidade do desenvolvimento de habilidades específicas e apontado todos os benefícios pedagógicos gerados pelo jogo de xadrez.
		
		\vspace{\onelineskip}
		
		\noindent
		\textbf{Palavras-chave}: Escola. Instrumento pedagógico. Educação.
	\end{resumoumacoluna}
	
	\textual
	
	\section*{Introdução}
	
	No setor escolar o jogo de xadrez é visto e empregado como um instrumento pedagógico que pode ajudar a despertar no aluno valores educacionais descuidados na educação neoliberal. Isto significa que, a prática do xadrez pode colaborar no desenvolvimento cognitivo do aluno, para Silva (2009) jogar xadrez é considerado por muitos uma atividade complexa, no entanto pode ser ensinado de formas simples, que irá ser assimilado após estudoscontínuos.
	
	CHRISTOFOLETT afirmar que:
	
	\begin{citacao}
		O xadrez é um fascinante jogo, que contém diversas combinações e estratégias, as quais necessitam de muito raciocínio e também de criatividade na elaboração eficaz, para se atingir o plano final de uma partida, o tão almejado xeque-mate. (CHRISTOFOLETT, 2007, p.42)
	\end{citacao}
	
	O objetivo desta pesquisa foi verificar quais os benefícios do jogo de xadrez e sua importância no processo ensino aprendizagem. A relevância dessa pesquisa é justificada pelas poucas pesquisas sobre o tema no estado e na região.
	
	\section*{Material e Método}
	
	Este artigo caracteriza-se por ser uma pesquisa de revisão bibliográfica considerando as contribuições de autores como SILVA (2009); Góes (2002); CHRISTOFOLETT (2007); SOUZA (2010), e que teve uma abordagem qualitativa, e usou a revisão de literatura como instrumento de coleta de dados. A revisão de literatura ocorreu entre outubro de 2015 a agosto de 2016 buscou os descritores em língua portuguesa: “Xadrez escolar”, “Benefícios do xadrez”, “Importância do xadrez pedagógico”, nos bancos de dados da Scielo, Biblioteca digital da USP, UNESP, UFSC entre outros.
	
	A partir desses estudos encontrados, realizou-se um rastreamento das investigações que abordassem o Xadrez no contexto escolar. As revisões tiveram como intuito buscar respostas para as questões da pesquisa, e com uma análise em profundidade de cada informação para descobrir possíveis incoerências ou contradições e utilizar fontes diversas, cotejando-ascuidadosamente.
	
	
	\section*{Resultados e Discussão}
	
	Conforme Souza (2016) o xadrez consegue ser uma dessas atividades extracurriculares que mantem o aluno envolvido de modo que o torne e o faça parte da construção de seu conhecimento.
	
	O fascínio do xadrez inspira no aluno a vontade de aprender, e este é o fator principal do aluno que se dispõe a jogar Xadrez. Pois para o aluno que estar aprendendo jogar xadrez é como brincar, e brincar é uma atividade indispensável para o desenvolvimento integral do aluno.
	
	O jogo de xadrez é uma atividade extremamente atrativa para os alunos, que descobrem rapidamente a sua beleza, e que podem ser compartilhadas com seus colegas, através de partidas, resolução de problemas e composições estéticas além de da interação com novas tecnologias.
	
	Nesse contexto a pesquisa tem o objetivo de avaliar o jogo como um todo e seus benefícios, Oliveira cita que:
	
	\begin{citacao}
Consideramos os jogos como um importante recurso tanto para a compreensão diagnóstica dos processos cognitivos desencadeados, quanto para a atuação pedagógica e psicopedagógica junto ao sujeito-autor de seu próprio conhecimento. Nós apoiamos na teoria piagetiana para a elaboração desse estudo que pretende verificar as relações ou as possíveis correspondências entre as condutas de escolares no jogo e construções da perspectiva espacial e social. (OLIVEIRA, 2005, p.15)

	\end{citacao}

O xadrez é um dos jogos capazes de promover a melhoria do processo ensino-aprendizagem.

Nas pesquisas de Silva (2009); Góes(2002) foi visto que, sendo o xadrez uma atividade conceituada e através da ligação de suas feições esportivas, artísticas, cognitivas e culturais e sua inserção na área educacional é muito reconhecida, devido a prática sistemática pelos alunos contribuindo assim positivamente para o desenvolvimento de habilidades, qualidades, através de seu caráter particularmente formador, essas são algumas das características do xadrez e suas implicações na educação:

\begin{table}[]
	\centering
	\caption{Adaptada pelo autor. Fonte: Silva (2009); Góes (2002).}
	\label{my-label}
	\begin{tabular}{|l|l|}
		\hline
		1.  & Desenvolvimento do autocontrolepsicofísico;                                                \\ \hline
		2.  & Avaliação da hierarquia do problema e a locação do tempodisponível;                        \\ \hline
		3.  & Desenvolvimento da capacidade para pensamento abrangente eprofundo;                        \\ \hline
		4.  & Empenho no progressocontinuo;                                                              \\ \hline
		5.  & Criatividade, imaginação e aprevisão;                                                      \\ \hline
		6.  & Respeito a opinião dointerlocutor;                                                         \\ \hline
		7.  & Capacidade para o processo de tomar decisões comautonomia;                                 \\ \hline
		8.  & Capacidade para o pensamento e execução lógicos, auto consistência e fluidez deraciocínio; \\ \hline
		9.  & Atenção e aconcentração;                                                                   \\ \hline
		10. & O julgamento e oplanejamento;                                                              \\ \hline
		11. & Aimaginação;                                                                               \\ \hline
		12. & Amemória;                                                                                  \\ \hline
		13. & A vontade de vencer, a paciência e oautocontrole;                                          \\ \hline
		14. & O espirito de decisão e acoragem;                                                          \\ \hline
		15. & A lógica matemática, o raciocínio analítico e asíntese;                                    \\ \hline
		16. & Ainteligência;                                                                             \\ \hline
		17. & Organização metódica do estudo e o interesse pelas línguasestrangeiras;                    \\ \hline
		18. & Raciocíniológico-matemático;                                                               \\ \hline
		19. & Visãoestratégica;                                                                          \\ \hline
		20. & Percepçãotemporal;                                                                         \\ \hline
		21. & Agilidade depensamento;                                                                    \\ \hline
		22. & Segurança em tomadas dedecisões;                                                           \\ \hline
		23. & APerseverança.                                                                             \\ \hline
	\end{tabular}
\end{table}

\FloatBarrier

Essas são algumas das diversas contribuições oriundas do aprendizado do jogo de xadrez no contexto escolar, são provenientes das transferências de aprendizagem, sendo esses alguns dos benefícios apontados pela bibliografia da área em relação ao aprendizado desta modalidade esportiva na escola.

CHRISTOFOLETTI	adaptou	uma	tabela	com	as	características	e	as implicações educacionais do xadrez, vejaabaixo:

\captionsetup{width=17cm}

	\begin{longtable}{|p{.45\linewidth}|p{.45\linewidth}|}
		\caption{Resumo da relação entre o xadrez e suas implicações nos aspectos educacionais. Fonte: CHRISTOFOLETTI (2007).}
		\label{my-label} 
		\endfirsthead
		\endhead
		\hline
		\textbf{Características do xadrez}                                                              & \textbf{Implicações nos aspectos educacionais}                                              \\ \hline
		Concentração.                                                                                   & Desenvolvimento do autocontrole psicofísico.                                                \\ \hline
		Fornecer um número de movimentos num determinado tempo.                                         & Avaliação da hierarquia do problema e a locação do tempo disponível.                        \\ \hline
		Movimentar peças após exaustiva análise de lances seguintes.                                    & Desenvolvimento da capacidade para pensamento abrangente e profundo.                        \\ \hline
		Encontrado um lance, a procura de outro melhor.                                                 & Empenho no progresso contínuo.                                                              \\ \hline
		Direcionar a uma conclusão brilhante uma posição aparentemente sem possibilidades (combinação). & Criatividade e imaginação.                                                                  \\ \hline
		O resultado indica quem tinha o melhor plano.                                                   & Respeito à opinião do interlocutor.                                                         \\ \hline
		Entre várias possibilidades, escolher uma única, sem ajuda externa.                             & Capacidade para o processo de tomar decisões com autonomia.                                 \\ \hline
		Um movimento deve ser consequência lógica do anterior antevendo o seguinte.                     & Capacidade para o pensamento e execução lógicos, auto consistência e fluidez de raciocínio. \\ \hline
	\end{longtable}

Ferreira Et al. Afirma que com aulas de xadrez, pode-se desenvolver e trabalhar diversas disciplinas da seguinte forma:
	\begin{citacao}
Artes: trabalho da confecção de peças com materiais reciclados.
Inglês: trabalho com palavras técnicas do xadrez, para poder se utilizar nos jogos on-line, onde o inglês é essencial.

Educação física: trabalhar a socialização e realizar torneios entre os alunos. Entre os torneios pode ser realizado o xadrez humano, onde os alunos representam peças de um tabuleiro de xadrez.

História: trabalhar a lenda da origem do xadrez e a evolução do xadrez com a evolução da sociedade, mostrando que as peças foram mudando com a evolução da sociedade.

Informática: jogar de forma on-line partidas e torneios como também trabalhar com programas específicos de xadrez.

Geografia: trabalhar com as características dos países que deram a origem ao xadrez como a Pérsia e a Arábia, bem como os países europeus que apresentaram importância para o desenvolvimento do jogo.

Português: trabalhar sobre a literatura no jogo do xadrez, culminando com uma amostra sobre poesias com o tema de xadrez.

Em uma perspectiva interdisciplinar, os alunos se sentem mais motivados, mais capazes de lidar com questões e problemas complexos, e mais engajados em pensamentos de nível mais alto. Eles aprendem a ver conexões e a lidar com a contradição. Mostram mais criatividade e atenção, e até mesmo, melhor assimilação em virtude das múltiplas conexões, além de ganhar perspectiva em relação às disciplinas. (FERREIRA, 2013, p.5)

	\end{citacao}
Confirmado assim algumas das possibilidades de multidisciplinaridade do xadrez.
	
	\section*{Conclusões}
	
	Este projeto foi de suma importância, pois com ele pode-se realmente entender as possibilidades que o xadrez pode proporcionar, principalmente na interdisciplinaridade, auxiliando nas habilidades como a atenção e a concentração, a capacidade de julgamento, a imaginação e a antecipação, a ativação da memória, a vontade de vencer trabalhando a paciência e o autocontrole, o espírito de decisão e a coragem, a lógica matemática, o raciocínio analítico e sintético, a criatividade, a inteligência, a organização metódica do estudo e o interesse pelas línguas estrangeiras, portanto a prática do xadrez na escola poderá trazer inúmeros benefícios.
	
	\section*{Instituição de Fomento}
	
	Instituto Federal de Rondônia – Campus Ariquemes.
	
	\section*{Referências}
	
	CHRISTOFOLETTI, D. F. A. O xadrez nos contextos do lazer, da escola e profissional: aspectos psicológicos e didáticos. 154 p. Dissertação, Mestrado em Ciências da Motricidade
	- Universidade Estadual Paulista Júlio de Mesquita Filho, Rio Claro, 2007.
	
	FEREIRA, R.M. Et al. O desenvolvimento de habilidades no estudo da matemática utilizando o xadrez como um recurso didático- pedagógico. Nova revista cientifica, Contagem, v. 2, n. 2, 1-14, 2013. Disponível em:
	<http://177.159.202.218:83/index.php/NOVA/article/view/67> Acesso em 27 jul. 2016.
	
	GÓES, D. C. O jogo de xadrez e a formação do professor de Matemática. 2002. 90f. Dissertação (Mestrado em Engenharia de Produção) - Programa de Pós-Graduação em Engenharia de Produção, UFSC, Florianópolis. Disponível em:
	<https://repositorio.ufsc.br/bitstream/handle/123456789/
	83221/190318.pdf?sequence=1\&isAllowed=y> Acesso em: 15 jul.2016.
	
	OLIVEIRA, F.N. Um estudo das interdependências cognitivas e sociais em escolares de diferentes idades por meio do jogo xadrez simplificado. Tese (Doutorado em Educação) - Universidade de Campinas, Faculdade de Educação. Campinas: 2005.
	
	SILVA, W. Raciocínio lógico e o jogo de xadrez: em busca de relações. Tese (Doutorado em Educação) - Universidade estadual de Campinas, Faculdade de Educação - Campinas: 2009.
	
	SOUZA, P. R. T. de. As possibilidades de aplicação do Jogo de Xadrez no processo Ensino-aprendizagem de Matemática. 2010. Revista Virtual P@rtes. V.00 p.eletrônica. Junho 2010. Disponível em: <http://www.partes.com.br/educacao/jogodoxadrez.asp>. Acessoem 19 ago. 2016.
	
	
\end{document}
