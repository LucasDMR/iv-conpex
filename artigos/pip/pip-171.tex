\documentclass[article,12pt,onesidea,4paper,english,brazil]{abntex2}

\usepackage{lmodern, indentfirst, color, graphicx, microtype, lipsum,textcomp}			
\usepackage[T1]{fontenc}		
\usepackage[utf8]{inputenc}		

\setlrmarginsandblock{2cm}{2cm}{*}
\setulmarginsandblock{2cm}{2cm}{*}
\checkandfixthelayout

\setlength{\parindent}{1.3cm}
\setlength{\parskip}{0.2cm}

\SingleSpacing

\begin{document}
	
	\selectlanguage{brazil}
	
	\frenchspacing 
	
	\begin{center}
		\LARGE SILAGEM DE COPRODUTO DA INDÚSTRIA DE PALMITO BABAÇU COMBINADA COM DIFERENTES ADITIVOS ABSORVENTES
		PARA ALIMENTAÇÃO DE RUMINANTES\footnote{Trabalho realizado dentro da Zootecnia (Nutrição e alimentação animal), com financiamento do DEPESP, Campus Ariquemes.}
		
		\normalsize
		Jaqueline C. Scaramussa\footnote{Bolsista (EM), jackyscaramussa@hotmail.com, Campus Ariquemes.} 
		Gabriel S. Stoinski\footnote{Bolsista (EM), gabrielstoinsk@gmail.com, Campus Ariquemes.} 
		Luciane C. Codognoto\footnote{Orientadora, luciane.codognoto@ifro.edu.br, Campus Ariquemes.} 
		Thassiane T. Conde\footnote{Co-orientadora, thassiane.conde@ifro.edu.br, Campus Ariquemes.} 
	\end{center}
	
	\noindent A utilização de resíduos agroindustriais para produção de silagens é comum em sistemas mais intensivos de produção de ruminantes, possibilitando sustentabilidade ao sistema de produção pecuária e retorno econômico, além de diminuir a disputa por produtos agrícolas de uso na alimentação humana e reduzir o acúmulo de potenciais poluentes ambientais. Coprodutos de indústrias, como a entrecasca
	/bainha de palmáceas descartados na fabricação de conservas de palmito, apresentam alto teor de umidade e requer a inclusão de aditivos para elevar os teores nutricionais e de matéria seca, beneficiando a fermentação da silagem. O objetivo foi avaliar o efeito de aditivos nutricionais na ensilagem de resíduo agroindustrial da conserva de palmito babaçu (gênero Orbignya). O coproduto foi ensilado adotando-se o delineamento experimental inteiramente casualizado, totalizando 24 silos/unidades experimentais (baldes vedados) adaptados com válvula tipo Bunsen, e compactados a densidade média de 500 kg m-3. Com base na matéria natural, os tratamentos constituíram-se em coproduto da industrialização de palmito babaçu (CIPB), processado em ensiladeira, e combinado com os aditivos: a) CIPB + 10\% de quirera de milho (QM); b) CIPB + 10\% de raspa de mandioca (RM);
	c) CIPB + 15\% de quirera de arroz (QA); d) CIPB + 5\% de QA; e) CIPB + 15\% de casca de café (CC); e, f) CIPB + 5\% de CC. Aos 86 dias de fermentação, os silos foram abertos, determinados a composição químico-bromatológica das silagens. Os tratamentos/silagens que foram incluídos os aditivos nutricionais 10\% de RM, 15 \% de QA e 15 \% de CC resultaram no aumento do teor de matéria seca das silagens (P<0,05), respectivamente, 30,48 \%, 30,67 \% e 31,40 \%, e média geral de 28,62\%. Entretanto, a silagem de CIPB combinada com aditivo a 10 \% de RM, apresentou o menor (P<0,05) teor de proteína bruta, equivalente a 5,46\%. Conclui-se que o coproduto da industrialização de conserva de palmito babaçu pode ser classificado como alimento de média qualidade e a inclusão dos aditivos possibilita o aumento do teor da matéria seca da silagem, melhorando o valor nutritivo da silagem produzida.
	
	\vspace{\onelineskip}
	
	\noindent
	\textbf{Palavras-chave}: Alimento alternativo. Matéria seca. Umidade.
	
\end{document}
